\documentclass[12pt]{fiche-rd-info}
\usepackage[utf8]{inputenc}
\usepackage[T1]{fontenc}

\begin{document}

\authorA{Pierre-Yves}{Hervo}
\authorB{Paul-François}{Jeau}

\begin{fichesuivi}{21 octobre 2013}{27 octobre 2013}
	\tempstravailA{10}{00}
	\tempstravailB{10}{00}

\paragraph{}
	\begin{travaileffectue}
		\begin{itemize}
			\item Lecture de 4 nouveaux articles : simple, réalisée à $100$ \% ;
			\item Recherche de nouveaux articles : simple, réalisée à $100$ \% ;
			\item Réunion avec M. Perreira Da Silva jeudi  : simple, réalisée à $100$ \% ;
			\item Point de réflexion sur la future orientation du projet  : moyen, réalisée à $100$ \% ;
		\end{itemize}
	\end{travaileffectue}

\paragraph{}
	\begin{echange}
		\begin{itemize}
			\item Nous avons rencontré M. Perreira Da Silva jeudi après midi afin de faire un premier point sur la phase bibliographique. Lors de cette réunion nous avons revu les idées tirées des articles lus. Ce qu'il ressort de cette réunion est la nécessité de déterminer l'orientation du projet. En effet nous avons pour le moment des articles qui traitent d'amélioration de photographie, de visages dans des photos, de critères esthétiques... Or il nous faut choisir une cible de travail, et savoir ce sur quoi nous concentrerons la fin de la phase bibliographique. Le but du projet peut se différentier sur une amélioration de photo de portait sur un critère très précis (à pousser), un ensemble de critères, ou sur un aspect qui relève plus de l'évaluation.
			\item Il est clair que nous n'effectuerons pas d'opérations modifiant la photo originale telle que l'objet serait méconnaissable au retour. D'où l'importance de déterminer la limite de la manipulation de la photo en entrée.
			\item Le but à l'issue de la réunion est donc de déterminer vers quel type d'amélioration nous voulons aller pour pouvoir rechercher les outils, méthodes, techniques existantes afin de pouvoir les comparer et in fine émettre des propositions pertinentes.
		\end{itemize}
	\end{echange}	
	
\paragraph{}
	\begin{planification}
		\begin{itemize}
			\item Établir les fiches des derniers articles plus globaux ;
			\item Déterminer l'orientation du projet avant le début de la semaine de rentrée ;
			\item Étudier des articles en correspondance avec la nouvelle orientation ; 
			\item Rechercher les outils traitant des problématiques relevées ;
			\item Contacter M. Perreira Da Silva pour lui faire part de notre choix ; 
		\end{itemize}
	\end{planification}
\end{fichesuivi}

\end{document}
