\documentclass[12pt]{fiche-rd-info}
\usepackage[utf8]{inputenc}
\usepackage[T1]{fontenc}

\begin{document}

\authorA{Pierre-Yves}{Hervo}
\authorB{Paul-François}{Jeau}

\begin{fichesuivi}{28 octobre 2013}{03 novembre 2013}
	\tempstravailA{1}{00}
	\tempstravailB{1}{00}

\paragraph{}
	\begin{travaileffectue}
		\begin{itemize}
			\item Nous avons discuté de la nouvelle orientation que souhaitions donner au projet de recherche. Nos premières lectures nous ont permis de cibler ce que nous voudrions traiter par la suite: moyenne, réalisée à $100$ \% ;
		\end{itemize}
	\end{travaileffectue}

\paragraph{}
	\begin{echange}
		\begin{itemize}
			\item Nous avons envoyé un mail à  M. Perreira Da Silva pour lui faire part de notre choix d'orientation. Les images en entrée seraient des “Photographies de type photo identité”
(avec prépondérance du visage). Les traitements qui nous intéressent
relèvent deux aspects:
- sur l’image de manière générale tels que liés à l’ Exposition,
Eclairage, Balance des blancs (traitements légers en vue de traitements
plus poussés sur le visage)
- sur le visage : comme la détection du visage, opérations de maquillage
pour masquer les signes tels que les rides, taches de rousseur, reflets,
acné. (Nous ne traiterons pas le cas des dents). Nous voulons aussi
étudier le cas du port d’objet type lunettes sur les traitements si cela a
déjà été réalisé.
Nous ne traiterons pas des opérations modifiant la structure du visage et
de la photographie	
		\end{itemize}
	\end{echange}	
	
\paragraph{}
	\begin{planification}
		\begin{itemize}
			\item La semaine de reprise nous allons centrer nos recherches sur les outils qui réalisent des améliorations automatique de photo de portrait  ; 
		\end{itemize}
	\end{planification}
\end{fichesuivi}

\end{document}
